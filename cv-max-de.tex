\documentclass{maxcv}

\usepackage[ngerman]{babel}

\name{Maximilian Neudert} % required
\location{Mannheim, Germany} % required
\email{max@neudert.me} % required
\phone{+49 16092957502} % required
\github{https://github.com/maxneuds} % optional
% \socials{https://linkedin.com/in/max} % optional

\begin{document}

\CVHeader{}

% ============================
% Experience
% ============================
\section{Technische Erfahrung}

\textbf{Programmiersprachen:} Python, R, SQL, C\#, Matlab, {\LaTeX}

\vspace{1 ex}

\begin{twocolentry}{}
    \textbf{Python Frameworks \& APIs}
\end{twocolentry}
\vspace{-1 em}
\begin{itemize}
    \setlength{\itemsep}{0pt}
    \item \textbf{Daten Verarbeitung:} Pandas, Polars, Numpy
    \item \textbf{Daten Visualisierung:} Plotly, Matplotlib, Seaborn
    \item \textbf{Web Frameworks:} FastAPI, Streamlit, Gradio
    \item \textbf{Machinelles Lernen:} Transformers, PyTorch, TensorFlow, Scikit-learn, XGBoost
    \item \textbf{Datenbanken:} SQLAlchemy, PyMongo
    \item \textbf{Verwaltung und Virtualisierung:} Poetry, Pipenv, Conda, Virtualenv, Pip
\end{itemize}\vspace{-\baselineskip}

\vspace{1 ex}

\textbf{R Daten \& Visualisierung:} dplyr, tidyr, fst, ggplot2, Plotly, Shiny

\vspace{1 ex}

\textbf{Datenbanken:} MongoDB, PostgreSQL, OracleDB, DynamoDB, Redshift

\vspace{1 ex}

\textbf{Infrastruktur CI/CD:} Docker, Git, Github, Github Actions, Jenkins, Jira, Confluence

\vspace{1 ex}

\begin{twocolentry}{}
    \textbf{Cloud Technologien}
\end{twocolentry}
\vspace{-1 em}
\begin{itemize}
    \setlength{\itemsep}{0pt}
    \item \textbf{AWS:} ECS, S3, Lambda, Bedrock, Route53, AppRunner, ALB
    \item \textbf{Azure:} AI Search, AI Foundry, OpenAI
\end{itemize}\vspace{-\baselineskip}

\section{Berufserfahrung}

\begin{twocolentry}{2024/01 \dash{} Jetzt}
    \textbf{Senior Data Scientist}, Continental Engineering GmbH \dash{} Frankfurt am Main, DE
\end{twocolentry}
\vspace{0.10 cm}
\begin{onecolentry}
    \begin{highlights}
        \item Leitung aller Data Science und Smart Application Projekte des \href{https://conti-engineering.com/solutions/consulting-support/data-services/}{Data Services Team} (20+ Mitarbeiter)
        \item Entwicklung und Einführung einer Live-Datenanalyseplattform zur Optimierung von Nachschubprozessen und des Shopfloor Managements zur Reduzierung von Stillständen und Steigerung der Effizienz
        \item Entwicklung einer Multi-Korrelationsanalyse zur Ursachenidentifikation in der Produktion
        \item Entwicklung eines Chatbots zur Analyse von EU-Vorschriften mit AWS Bedrock
        \item Mitentwicklung einer KI-Plattform für das Anforderungsmanagement mit Feintuning und Deployment von Large Language Models sowie klassischem Machine Learning
        \item Skalierung und Migration eines Marktanalyse- und Planungstools mit Datenbank und Dashboard in die AWS Cloud
        \item Entwicklung von MES Performance Dashboards für CPK und MSA Analysen
        \item Optimierung von ETL-Prozessen und Datenpipelines für maximale Performance und Skalierbarkeit
        \item Erstellung von Best Practices und Richtlinien für Data Science Projekte
        \item Entwicklung datengetriebener Projektarchitekturen sowie CI/CD-Pipelines und deren Umsetzung
        \item Entwicklung einheitlicher Workflows auf Basis containerisierter Entwicklungs und Deployment Prozesse
        \item Durchführung von Code- und Projekt-Reviews sowie Mentoring und Coaching von Junior Data Scientists
    \end{highlights}
\end{onecolentry}

\vspace{0.2 cm}

\begin{twocolentry}{2021/01 \dash{} 2023/12}
    \textbf{Data Scientist}, Continental Engineering GmbH \dash{} Frankfurt am Main, DE
\end{twocolentry}
\vspace{0.10 cm}
\begin{onecolentry}
    \begin{highlights}
        \item Leitung der Entwicklung eines Marktanalyse- und Planungstools mit Datenbank und Dashboard
        \item Einführung containerisierter Entwicklungs und Deployment Workflows zur Standardisierung von CI/CD-Prozessen
        \item Migration von Data Science und Service Anwendungen in die Cloud zur Verbesserung von Skalierbarkeit und Verfügbarkeit
        \item Konfiguration und Verwaltung von Linux Servern für das Hosting von Anwendungen
        \item Entwicklung einer Dokumentenanalyseplattform mit Machine Learning gestützter Dokumenten Clusterbildung
        \item Unterstützung bei der Entwicklung einer Logistik und Supply Chain Management Plattform für Allokations- und Produktionsplanung während der Corona Krise, zur Ermöglichung fairer Produktverteilung und optimierter Materialnutzung in Zeiten von Knappheit
    \end{highlights}
\end{onecolentry}

% ============================
\clearpage
% ============================

\vspace{0.2 cm}

\begin{twocolentry}{2020/05 \dash{} 2020/10}
    \textbf{Masterand}, Continental AG \dash{} Frankfurt am Main, DE
\end{twocolentry}
\vspace{0.10 cm}
\begin{onecolentry}
    \begin{highlights}
        \item Thesis: ``Data Mining on Sensor Data for Industrial Engineering''
        \item Entwicklung eines belastbaren Proof of Concept, der Sensoranalysen in umsetzbare Erkenntnisse für industrielle Fertigungssysteme umwandelt
    \end{highlights}
\end{onecolentry}

\vspace{0.2 cm}

\begin{twocolentry}{2017/11 \dash{} 2020/04}
    \textbf{Werkstudent}, Software AG \dash{} Darmstadt, DE
\end{twocolentry}
\vspace{0.10 cm}
\begin{onecolentry}
    \begin{highlights}
        \item Durchführung von Produkttests und Erweiterung des Testpools für das Produkt Natural für Windows mit dem Tool Silk Test von Micro Focus sowie für die Produkte Natural und Predict unter Linux
        \item Modernisierung und Automatisierung des Testpools auf Windows und Linux inklusiv der Umstellung des bestehenden Testpool-Scripting von bash, batch, csh und Perl auf Python
        \item Automatisierung der Ausführung des Testpools mit Jenkins
        \item Erstellung der kompletten, technischen Dokumentation des Testpools
    \end{highlights}
\end{onecolentry}

\vspace{0.2 cm}

\begin{twocolentry}{2013/04 \dash{} 2014/09}
    \textbf{Übungsleitung}, Technische Universität Darmstadt \dash{} Darmstadt, DE
\end{twocolentry}
\vspace{0.10 cm}
\begin{onecolentry}
    \begin{highlights}
        \item Übungsleitung für Mathematik I und II für Maschinenbau, Einführung in die Stochastik und Geometrie für das Lehramt
    \end{highlights}
\end{onecolentry}


% ============================
% Education
% ============================
\section{Bildung}

\begin{twocolentry}{2018/10 \dash{} 2020/09}
    \textbf{Hochschule Darmstadt}, \href{https://h-da.de/en/studies/study-programmes/study-programmes/natural-science-and-mathematics/data-science}{M.Sc. Data Science}
\end{twocolentry}
\vspace{0.10 cm}
\begin{onecolentry}
    \begin{highlights}
        \item Grade:\@ 1.8
        \item \textbf{Coursework:} Explorative Datenanalyse und Visualisierung, Computerintersive Methoden,\\
        Big Data Technologien, Nichtlineare und Nichtparametrische Modelle, Text- und Web Mining,\\
        Business Intelligence, Angewandte Künstliche Intelligenz, Gemischt-Ganzzahlige Optimierung
    \end{highlights}
\end{onecolentry}

\vspace{0.2 cm}

\begin{twocolentry}{2017/10 \dash{} 2018/07}
    \textbf{Technische Universität Darmstadt}, M.Sc. Computational Engineering
\end{twocolentry}
\vspace{0.10 cm}
\begin{onecolentry}
    \begin{highlights}
        \item unvollständig
        \item Fokus auf Robotik, Computer Vision und Maschinelles Lernen
    \end{highlights}
\end{onecolentry}

\vspace{0.2 cm}

\begin{twocolentry}{2015/10 \dash{} 2017/07}
    \textbf{Technische Universität Darmstadt}, M.Sc. Mathematics
\end{twocolentry}
\vspace{0.10 cm}
\begin{onecolentry}
    \begin{highlights}
        \item unvollständig
        \item Vertiefungen in Analysis, Numerische Mathematik und Geometrie
    \end{highlights}
\end{onecolentry}

\vspace{0.2 cm}

\begin{twocolentry}{2011/10 \dash{} 2015/07}
    \textbf{Technische Universität Darmstadt}, B.Sc. Mathematics
\end{twocolentry}
\vspace{0.10 cm}
\begin{onecolentry}
    \begin{highlights}
        \item Grade:\@ 2.7
        \item Nebenfach in Wirtschaftswissenschaften und Spezialisierung in Mathematik
    \end{highlights}
\end{onecolentry}

% ============================
% Languages
% ============================
\section{Languages}

\begin{highlights}
    \item Deutsch \dash{} Muttersprachlich
    \item English \dash{} Verhandlungssicher (C1)
\end{highlights}

% ============================
% Projects
% ============================
\section{Projects}

% \begin{twocolentry}{Private}
%     \textbf{Unnamed Mobile Game}
% \end{twocolentry}
% \vspace{0.10 cm}
% \begin{onecolentry}
%     \begin{highlights}
%         \item Development of a mobile game using Unity and \csharp{}\\
%         \SkillTagsInline{Unity, \csharp{}, Git}
%     \end{highlights}
% \end{onecolentry}

% \vspace{0.2 cm}

% \begin{twocolentry}{\href{https://github.com/maxneuds/hsr-farmer}{github.com/hsr-farmer}}
%     \textbf{Android Game Bot}
% \end{twocolentry}
% \vspace{0.10 cm}
% \begin{onecolentry}
%     \begin{highlights}
%         \item Developed a bot for monster hunting in an android game\\
%         \SkillTagsInline{Python, OpenCV, Numpy, ADB, Docker}
%     \end{highlights}
% \end{onecolentry}

% \vspace{0.2 cm}

% \begin{twocolentry}{\href{https://github.com/maxneuds/poketrader}{github.com/poketrader}}
%     \textbf{Android Game Bot}
% \end{twocolentry}
% \vspace{0.10 cm}
% \begin{onecolentry}
%     \begin{highlights}
%         \item Developed a bot for monster trading in an android game\\
%         \SkillTagsInline{Python, OpenCV, Tesseract, Numpy, ADB, Docker}
%     \end{highlights}
% \end{onecolentry}

% \vspace{0.2 cm}

\begin{twocolentry}{}
    \textbf{Homelab}
\end{twocolentry}
\vspace{0.10 cm}
\begin{onecolentry}
    \begin{highlights}
        \item Homelab mit eigenem Routing, Firewall und NAS
        \item Private Cloud mit OPNsense, Docker und Virtualisierung
        \item Entwicklung von KI-Tools und -Anwendungen für persönliche Projekte
    \end{highlights}
\end{onecolentry}

% ============================
% Technical Skills
% ============================

\SkillSection{Preferred Tech Stack}{Linux, Docker, VS Code, Python, FastAPI, MongoDB, AWS}

\SkillSection{Tech Exeperience Tags}{
    AWS, Azure, Docker, OracleDB, MongoDB, PostgreSQL, Pandas, Polars, FastAPI, SQLAlchemy, Seaborn, XGBoost, Numpy, Scikit-learn, PyTorch, TensorFlow, OpenCV, Matplotlib, Plotly, Streamlit, Gradio, OpenAI, Ollama, HuggingFace, Transformers, Bedrock, Lambda, Angular, Jira, Confluence, \LaTeX, Git, Linux
}

% \SkillSection{Programming Languages}{Python, SQL, JavaScript, \csharp{}}


\end{document}